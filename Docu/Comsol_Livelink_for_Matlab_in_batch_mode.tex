\documentclass[12pt]{scrartcl}%{article} % Beginn der LaTeX-Datei

%% twocolumn

\usepackage{amsmath,amssymb}  % erleichtert Mathe 
\usepackage{enumerate}% schicke Nummerierung

\usepackage{graphicx} % für Grafik-Einbindung
%\usepackage{hyperref}

\usepackage[ngerman]{babel}
\usepackage[T1]{fontenc}
\usepackage{lmodern}
 % Einstellungen, wenn man deutsch schreiben will, z.B. Trennregeln
\usepackage[utf8]{inputenc}  % für Unix-Systeme
  % ermöglicht die direkte Eingabe von Umlauten und ß
  % evt. obige Zeile ersetzen durch
  % \usepackage[ansinew]{inputenc}  % für Windows
  % \usepackage[applemac]{inputenc} % für den Mac

\usepackage{booktabs}

%%%%%%%%%%%%%%%%%%%%%%%%%%%%%%%%%%%%%%%%%%%%%%%%%%%%%%%%%%%%%%%%%%
%
%  ntheorem
%
\usepackage[thmmarks,amsmath,hyperref,noconfig]{ntheorem} 
  % erlaubt es, Sätze, Definitionen etc. einfach durchzunummerieren.
\newtheorem{satz}{Satz}[section] % Nummerierung nach Abschnitten
\newtheorem{hilfssatz}[satz]{Hilfssatz}
\newtheorem{kor}[satz]{Korollar}

\theorembodyfont{\upshape}
\newtheorem{beispiel}[satz]{Beispiel}
\newtheorem{bemerkung}[satz]{Bemerkung}
\newtheorem{definition}[satz]{Definition} %[section]

\theoremstyle{nonumberplain}
\theoremheaderfont{\itshape}
\theorembodyfont{\normalfont}
\theoremseparator{.}
\theoremsymbol{\ensuremath{_\blacksquare}}
\newtheorem{beweis}{Beweis}
\qedsymbol{\ensuremath{_\blacksquare}}
%\theoremclass{LaTeX}
%
% Ende ntheorem
%
%%%%%%%%%%%%%%%%%%%%%%%%%%%%%%%%%%%%%%%%%%%%%%%%%%%%%%%%%%%%%%%%%%


%\pagestyle{empty}
%
% Ändern der bedruckten Fläche der Seite
% \addtolength{\textwidth}{3cm}  % Befehl mit zwei Argumenten
% \addtolength{\textheight}{3cm}
% \hoffset-2cm % verschiebt das Textfenster nach links
% \voffset-5mm % verschiebt das Textfenster nach oben
%
%\parindent=0pt %% keine Einzug zu Beginn von Abs\"atzen
%\parskip=2mm   %% erzeugt einen zusätzliche Zeilenabstand zwischen
                %% Absätzen. Nötig bei \parindent=0pt


%%%%%%%%%%%%%%%%%%%%%%%%%%%%%%%%%%%%%%%%%%%%%%%%%%%%%%%%%%%%%%%%%%
% ermöglicht, farbigen Text zu drucken.
\usepackage{color}
% Und nun werden die Farben definiert - daran können Sie nach Belieben selber rumspielen.
\definecolor{white}{rgb}{1,1,1}
\definecolor{darkred}{rgb}{0.3,0,0}
\definecolor{darkgreen}{rgb}{0,0.3,0}
\definecolor{darkblue}{rgb}{0,0,0.3}
\definecolor{pink}{rgb}{0.78,0.09,0.51}
\definecolor{purple}{rgb}{0.28,0.24,0.55}
\definecolor{orange}{rgb}{1,0.6,0.0}
\definecolor{grey}{rgb}{0.4,0.4,0.4}
\definecolor{aquamarine}{rgb}{0.4,0.8,0.65}


\DeclareMathOperator{\GL}{GL} % einige Macro, siehe am Ende Abschn. 2
\newcommand{\N}{\mathbb{N}}
\newcommand{\Z}{\mathbb{Z}}
\newcommand{\Q}{\mathbb{Q}}
\newcommand{\R}{\mathbb{R}}
\newcommand{\C}{\mathbb{C}}
\newcommand{\cP}{{\mathcal P}} 

\begin{document}

\author{}
\title{COMSOL LiveLink für Matlab auf dem bwUniCluster}
\date{} %hier können Sie ein Datum eingeben, auch leer, sonst wird es
         %automatisch erzeugt

\maketitle % erzeugt den Kopf

\section{Allgemeines}

\begin{itemize}
\item Voraussetzung: Registrieren für den bwUniCluster und bwFileStorage auf der Website des KIT 
\item Verbindung aufbauen:
\begin{enumerate}
\item bwUniCluster  \\ \texttt{ssh -X präfix\_login@bwunicluster.scc.kit.edu}
\item bwFileStorage \\ \texttt{ssh präfix\_login@bwfilestorage-login.lsdf.kit.edu}
\end{enumerate}
\item Dateien vom lokalen PC in das Home-Verzeichnis des bwUniClusters kopieren:
\begin{enumerate}
\item Übertragung der lokalen Dateien auf den bwFileStorage Server: \\ \texttt{scp -r <local\_path> präfix\_login@bwfilestorage.lsdf.kit.edu:<remote\_path/>}
\item Übertragung der Dateien vom bwFileStorage auf den bwUniCluster
\begin{enumerate}
\item[I)] Anmeldung auf dem bwUniCluster
\item[II)] Kopieren der Dateien \\ \texttt{rdata 'cp -r \$BWFILESTORAGE/<path> \$HOME/<path>'}
\end{enumerate}
\end{enumerate}
\item Dateien vom bwUniCluster auf lokalen PC kopieren
\begin{enumerate}
\item Übertragung vom bwUniCluster zum bwFileStorage \\ \texttt{rdata 'cp -r \$HOME/<path> \$BWFILESTORAGE/<path>'}
\item Übertragung vom bwFileStorage Server auf lokalen PC \\ \texttt{scp präfix\_login@bwfilestorage.lsdf.kit.edu:<remote path> /<local path>}
\end{enumerate}
\item Batch-System des bwUnicluster
\begin{itemize}
\item Job-Submission: \texttt{msub}
\item Job-Canceling: \texttt{mjobctl -c <Job\_ID>}
\item Status der Batch-Jobs anzeigen: \texttt{showq}
\end{itemize}
\item Dinge die vor dem ersten Batch-Job erledigt werden müssen
\begin{itemize}
\item Der COMSOL-Server muss einmal manuell mit dem Befehl \texttt{comsol server matlab} gestartet werden um einen persönlichen Benutzernamen und ein Kennwort festzulegen.
\item COMSOL muss einmal manuell geöffnet werden und es muss bei \textit{Preferences > Security > Allow external processes and libaries} ein Haken gesetzt werden.
\end{itemize}
\end{itemize}

\section{Interactive Batch-Jobs}
Interaktive Berechnungen auf der Login-Node des bwUniClusters sind nicht erlaubt. Für interaktive Berechnungen muss auf interaktive Batch-Jobs (Option \texttt{-I}) zurückgegriffen werden. Beispiel:\\
\hspace*{0.5cm} \texttt{musb -I -V -l nodes=1:ppn=4 -l walltime=00:00:20:00} \\
Die restlichen Optionen können analog zu normalen Batch-Jobs gewählt werden.\\
Anschließend darf der Terminal nicht geschlossen werden und man muss darauf warten, dass einem die entsprechenden Ressourcen zugewiesen werden.

\section{bwUniCluster unter Windows}
\begin{itemize}
\item Zugriff über Putty mit den Einstellungen\\
\hspace*{0.5cm}\texttt{Host Name : präfix\_login@bwunicluster.scc.kit.edu}\\
\hspace*{0.5cm}\texttt{Port: 22}\\ 
\hspace*{0.5cm}\texttt{Connection Type: SSH} \\ 
Für graphische Anwendungen (z.B. Textdatei anzeigen) muss zusätzlich Xming installiert werden und in den Putty Einstellungen bei \textit{SSH > X11 > Enable X11 forewarding} ein Haken gesetzt werden.\\
Wenn man sich anschließend mit dem bwUniCluster verbindet so öffnet sich ein Terminal in dem man dann die selben Befehle verwenden kann wie beim arbeiten unter Linux.
\item Alternativ kann man sich auch über WinSCP auf dem bwUniCluster einloggen (Zur Verwaltung des Home-Verzeichnisses).
\end{itemize}

\section{Linux Terminal}
\begin{tabular}{ll} 
\toprule
\textbf{Linux Terminalbefehle}\\  
\midrule 
Nutzen & Befehl\\ 
\midrule 
Dateinmanager öffnen & \texttt{nautilus}\\
Verzeichnis wechseln & \texttt{cd <Verzeichnisname>}\\
In nächst höheres Verzeichnis wechseln & \texttt{cd ..}\\
Dateien Löschen & \texttt{rm -rf <Dateiname>} \\
\bottomrule
\end{tabular}

\end{document}

